\documentclass[11pt,a4paper]{moderncv}

% moderncv themes
%\moderncvtheme[blue, roman]{casual}
% optional argument are 'blue' (default), 'orange', 'red', 'green', 'grey' and 'roman' (for roman fonts, instead of sans serif fonts)
\moderncvtheme[blue, roman]{classic}                % idem

\usepackage[utf8]{inputenc}

% adjust the page margins
\usepackage[scale=0.8]{geometry}
\recomputelengths                             % required when changes are made to page layout lengths

% personal data
\firstname{Kai}
\familyname{Cao}
\mobile{+86 13818141545}
\email{ck89119@gmail.com}
%% \quote{``Do what you fear, and the death of fear is certain.''\\-- Anthony Robbins}

%\phone{(312) 413-8265}                      % optional, remove the line if not wanted
%\fax{312 996 1491}                          % optional, remove the line if not wanted
%\photo[64pt]{avatar.jpg}                         % '64pt' is the height the picture must be resized to and 'picture' is the name of the picture file;

\nopagenumbers{}                             % uncomment to suppress automatic page numbering for CVs longer than one page

%% \renewcommand*{\sectionfont}{\LARGE\sffamily\monospace\slshape}
%% \renewcommand*\addressfont{\fontfamily{pzc}\selectfont}
%% \renewcommand*\sectionfont{\fontfamily{pzc}\fontsize{20}{24}\selectfont}

\begin{document}
\maketitle

\section{Education}
\cventry{2012.9-2016.3}{M.E. in Computer Science}{Shanghai Jiao Tong University}{}{}{}
\cventry{2008.9-2012.6}{B.E. in Software Engineering}{Soochow University}{}{}{}

%\section{Master thesis}
%\cvline{title}{\emph{Title}}
%\cvline{supervisors}{Supervisors}
%\cvline{description}{\small Short thesis abstract}

\section{Community}
\cventry{GitHub}{github.com/ck89119}{}{}{}{}

\section{Project Experience}
\cventry{2015.4-2015.11}
{Test Suite}
{Shell, Python}{}
{CooTek}
{
It is a full test suite for CooTek keyboard engine on Linux platform.
Our engine seperates western languages and Chinese apart in function, so this test suite has two parts as well.
\newline My contributions in western language(English, French, German and so on, about 11 languages in total) part,
\newline 1.Wrote scripts to generate correction-test cases from user input data which was collected by my teammate.
\newline 2.Wrote scripts to generate exact-word-test cases for every unigram/bigram/trigram in our dictionaries.
\newline 3.Automatized western language test part and integrated it into test suite.
\newline My contributions in Chinese(Pinyin, Zhuyin, Cangjie, Bihua and Wubi) part,
\newline 1.Wrote scripts to generate exact-word-test cases for every word in our dictionaries.
\newline 2.Integrated scripts which were used for testing average rank, first rank percentage and sentence percentage.
\newline 3.Automatized Chinese test part and integrated it into test suite.
\newline
}

\cventry{2015.4-2015.11}
{Performance Test in an Android app}
{Java, Android}{}
{CooTek}
{
In order to simulate the real condition, we wrote an Android app for testing. My contributions,
\newline 1.Wrote a time testing module and a series derivative testing modules.
\newline
}

\cventry{2014.7–2014.8}
{Opcode Counting}
{Dalvik VM, Android}
{Intern}
{Alibaba}
{
The purpose of this project was to count the times each opcode is called when an Android app runs. My contributions,
\newline 1.Designed the whole architecture.
\newline 2.Determined the counting timing and the position of counting field.
\newline 3.Figured out the idea, switching to ALT-mode forcely, to optimize counting process.
\newline 4.Output the counting result by exploiting USER\_SIGNAL.
\newline
}

\cventry{2012.9–2015.3}
{LoCCS Power Analysis Platform}
{JNA, Side Channel Attack(SCA)}
{Research}{}
{
This project was a part of National ’973’ Project in Lab of Cryptology and Computer Security (LoCCS). This platform was designed to cover all operations during an SCA. My contributions,
\newline 1.Designed the origin UI of the software and implemented it in Java.
\newline 2.Tested JNA and JNI for Java calling C++ libraries.
\newline 3.Wrote a series of analytic modules of DES algorithm, e.g. CorrelationAnalysis, KnownKeyAnalysis, SecondOrderAnalysis.
\newline
}

\cventry{2011.12–2012.5}
{Research on constructing a binary tree efficiently}
{Algorithm}
{Research}{}
{
This was my graduation project for undergraduate study. The purpose of this project was to find efficient algorithms that could construct a binary tree from the two kinds of traversal sequence and information of each node. My contributions,
\newline 1.Found a constructing algorithm from postorder traversal and the first child of each node.
\newline 2.Found a constructing algorithm from postorder traversal and nextsibling of each node.
\newline 3.Created random binary trees(about 5000 nodes) for testing.
\newline
}

%\section{Awards}
%\cventry{2012}{Second Grade Scholarship for Graduates of Shanghai Jiaotong University}{}{}{}{}
%\cventry{2010}{Second Place in Programming Competition of Soochow University}{}{}{}{}
%\cventry{2009}{Honor Reward of ACM/ICPC Asian Regional in Wuhan}{}{}{}{}
%\cventry{2009}{First Grade Scholarship of Soochow University}{}{}{}{}
%\cventry{2009}{Zhu Jingwen Scholarship}{}{}{}{}
%\cventry{2007}{Second Prize in National Olympiad in Informatics in Provinces}{}{}{}{}

\section{Skills}
\cventry{Language}{C/C++, Java, Python, Shell, Haskell(little)}{}{}{}{}
\cventry{OS}{Windows, Linux, OS X(daily-work)}{}{}{}{}
\cventry{Version control}{Git, Svn}{}{}{}{}
\cventry{Android}{VM, Java Dev}{}{}{}{}
\cventry{Other}{Algorithm/Data Structure(enthusiastic), Vimer$\sim$}{}{}{}{}
%\cvline{}{C/C++, Java, Python, Shell, Haskell(little), Git/Svn, Algorithm/Data Structure(enthusiastic), Linux, OS X(daily-work), Vimer$\sim$}

%\section{Self-assessment}
%\cvline{}
%{
%I have huge passion for coding, math and algorithm, and I believe my programming skills will make our lives better and more convenient in the future. I have a very low laugh-point:) and I can work well under pressure. At the same time, I am a good team player and have great communication skills. I would greatly appreciate the chance to work at your company.
%}

%\section{Extra 1}
%\cvlistitem{Item 1}
%\cvlistitem{Item 2}
%\cvlistitem[+]{Item 3}            % optional other symbol

%\section{Extra 2}
%\cvlistdoubleitem[\Neutral]{Item 1}{Item 4}
%\cvlistdoubleitem[\Neutral]{Item 2}{Item 5}
%cvlistdoubleitem[\Neutral]{Item 3}{}

\end{document}
