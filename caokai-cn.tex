\documentclass[11pt,a4paper]{moderncv}

% moderncv themes
%\moderncvtheme[blue]{casual}                 % optional argument are 'blue' (default), 'orange', 'red', 'green', 'grey' and 'roman' (for roman fonts, instead of sans serif fonts)
\moderncvtheme[green]{classic}                % idem
\usepackage{xunicode, xltxtra}
\XeTeXlinebreaklocale "zh"
\widowpenalty=10000

%\setmainfont[Mapping=tex-text]{文泉驿正黑}

% character encoding
%\usepackage[utf8]{inputenc}                   % replace by the encoding you are using
\usepackage{CJKutf8}

% adjust the page margins
\usepackage[scale=0.8]{geometry}
\recomputelengths                             % required when changes are made to page layout lengths
\setmainfont[Mapping=tex-text]{Hiragino Sans GB}
\setsansfont[Mapping=tex-text]{Hiragino Sans GB}
\CJKtilde

% personal data
\firstname{程纯}
\familyname{}
%\title{}               % optional, remove the line if not wanted
\mobile{15601794167}                    % optional, remove the line if not wanted
\email{ccmicky89@gmail.com}                      % optional, remove the line if not wanted
%% \quote{\small{``Do what you fear, and the death of fear is certain.''\\-- Anthony Robbins}}

\nopagenumbers{}

\begin{document}
\maketitle

\section{教育}
\cventry{2008--2012}{本科}{苏州大学-计算机科学与技术学院}{}{}{}

\section{工作经历}
\cventry{2012.7--2013.11}{IBM}{}{}{}{}
\cventry{2013.12--2015.8}{同程旅游网络科技有限公司}{}{}{}{}
\cventry{2015.11--2017.7}{纽海信息科技有限公司-1号店}{}{}{}{}
\cventry{2017.8--至今}{达疆网络科技有限公司-达达}{}{}{}{}

\section{项目经历}
\cventry{2018.2-2018.5}
{工服自动识别系统}
{Tensorflow, Python}{}
{达达}
{
背景:为了规范骑士穿工服,引入了自拍上传人工审核机制 ,但是人工审核人力成本高且审核效率低下,所以为了提高审核人员的工作效率,采用机器初审筛选的方式。
\newline
\newline  我在整理生成训练数据方面的工作:
\newline 1.为了支持不同logo(京东到家和不同时期的达达)的工服,我对训练数据分类,用可视化程序辅助完成做了多套工服的数据标注和自动xml文件生成。
\newline 2.由于拍摄的背景,角度,光线,穿戴的遮挡,图案logo的扭曲,我也对不同场景下的数据抽样,生存训练数据。
\newline 3.自动化西方语言测试,并将其集成到测试框架中。
\newline
\newline 我在模型生成和线上部署的工作:
\newline 1.我们采用了基于tensorflow框架下的faster-rcnn模型,使用vgg-16网络结构和预训练好的模型。
\newline 2.线上cpu线下gpu的tf环境部署,自定义c++ engine库的roi-pool operation的多环境编译和安装。
\newline
\newline 取得成果:
\newline 1.可以对用户上传的照片在当天完成全量的审核结果,现在的机器审核一次通过率为80%。
\newline 2.在审核量一致情况下,对比直接人工规则抽取照片由人工审核和先由机器审核,人工只审核机器判定为不过的照片,加入机器审核的那一方可以找到的不符合要求的照片是人工规则直接抽取审核的两倍。
\newline
}

\cventry{2018.4-2018.7}
{人脸1:1识别,判断是否为本人}
{Tensorflow, Python}{}
{达达}
{
为了规范达达骑士,减少注册者与实际工作者非同一个人。
\newline
\newline 我的工作:
\newline 1.使用CASIA-web face数据库的10万张人脸图片为训练数据。
\newline 2.使用facenet 模型的inception-resnet-v1的网络结构和softmax的损失函数训练判别是否为同一个的模型,模型的auc=0.87。
\newline 取得成果:能有效的初筛出一批有问题的非本人照片,为后期的人工复审提供候选集。
\newline
}

\cventry{2017.8-2018.5}
{接单时间预估}
{Sklearn, Python}{}
{达达}
{
C端用户下单前后分别给出预计订单被接的时间。
\newline
\newline  我在特征工程方面的工作:
\newline 1.运力和订单空间分布特征通过一层全连接转化成一个特征向量。
\newline 2.地理位置特征:将历史难送(超时,未接单率高)的地址找出来,将相应地区的geohash id作为one hot特征。
\newline 3.将除了运力和订单空间分布特征之外,全部采用one-hot编码,通过gdbt做特征组合。
\newline 4.多分类数据极度不平衡,经过各种平衡采样测试,采用log(counts)的方式,在各个预测区间内的误差和总体误差效果都比较好。
\newline 
\newline  我在模型训练方面的工作:
\newline 1.标注数据通过历史数据将接单时间在0到30分钟内的数据向上取整,大于10分钟的归纳到10分钟内,生成标注数据。
\newline 2.通过sklearn的logistic regression multi class 完成多分类模型。
\newline
\newline  我在分流测试和数据分析方面的工作:
\newline 1.下单后预计接单时间在前台展示上有倒计时,正计时,和直接给出预计接单时间等不同展示方式,统计不同展示方式下的用户真实等待时长。
\newline 2.分析预计接单时间偏长偏短对用户下单,取消的影响。
\newline
\newline  取得成果:
\newline  1.全国的绝对误差时长从原来的取区域平均值(5min)下降到(100s)。
\newline  2.对用户的下单率和取消率在数据分析后感觉用户在dada的c端平台并不明显,没有显著改善。
\newline
}

\cventry{2017.11-2018.5}
{智能地址解析}
{Python}{}
{达达}
{
在微信公众号端通过给出的地址姓名电话等字符串,自动解析填表。
\newline  我的工作:
\newline 1.第一版本通过正则表达式,完成自动解析。
\newline 2.第二版本尝试LSTM+CRF完成自动解析,正在进行中。
\newline 
}

\cventry{2016.3-2017.7}
{SEM自动化投词}
{Sklearn,c++ DP, Python,Spark}{}
{1号店}
{
这是通过在自动投词引擎平台对代理投词的历史数据学习后,在一定预算条件下,配合运营目标,按天给出的离线投词引擎。
\newline  我的工作:
\newline 1.扩词:通过清洗站内搜索日志,对站内的高频搜索短语整理,补充投词词库,通过百度 扩词工具扩词。
\newline 2.通过词(短语)one-hot 向量化,将已有订单的词作为学习样本,通过Bayes算法将词 分到与商品相对应的具体类目,配合广告创意的模板套用,提高语意相关性(质量得分) 和投放的预算划分。
\newline 3.确定词的单次点击竞价cpc:计算每个sem投词的曝光/点击/订单数与cpc(单次点击价 格的)皮尔森相关性系数R(cos值),根据该系数决定cpc出价。
\newline 4.预估关键词效果:通过回归模型(svr)对历史曝光点击数据的学习,预测具体日期下, 关键的日曝光量,日点击量,日销售额。
\newline 5.给出推词方案:通过DP的简单背包算法,在一定预算条件下,最大化某个目标(点击, 曝光,销售额),给出自动化的出词方案。
\newline
}

\cventry{2015.12-2016.7}
{dis站内广告栏位推荐}
{Python, Spark}{}
{1号店}
{
通过选人和选品系统,对人群做出划分,选出推荐的候选商品,通过逻辑回归拟合CTR,给特定人群的广告栏位推荐商品。
\newline  我的工作:
\newline 1.人群划分:通过用户在指定时间段内的站内行为和该行为所在商品的类目,年龄,消费 水平,地域将用户划分为指定个数的群体。
\newline 2.人群(request),商品(response)特征选取,通过GBDT特征交叉,生成学习和预测数据。
\newline 3.通过sgdLR 训练模型,根据预测结果,取top5推送到广告栏位。
\newline
}

\cventry{2016-2017}
{关键词与商品的相关性计算接口}
{Word2vec c++ source}{}
{1号店}
{
这个接口给1号店商城商家提供自定义站内关键词,建立关键词竞价体系,提供实时计算关键词与商品相关性计算的接口。
\newline 1.通过word2vec模型,对分词建立词向量,词向量相加得到短句向量,计算商品与关键 词(句)间的相关性。
\newline
}

\cventry{2016-2017}
{dsp}
{Python, Spark, Sparkstreaming}{}
{1号店}
{
这个项目的目的是给第三方dsp公司提供基于1号店平台用户购买意图的预测,向dsp接口实时提供用户与商品的mapping关系。
\newline  我的工作:
\newline 1.spark-streaming,时间片间隔2min,一个yhd用户访问1号店2min后,根据其浏览行为,向dsp公司接口推送该用户的guid和更新或者建立他的用户意图预测下的10个商品id 的mapping关系。
\newline 2.给出用户对每个商品意图的打分: 根据用户的站内行为通过分别通过LR+GBDT和FM预测用户对每个商品的购 买意图,然后取top5推送,同时通过 出价上线*意图得分*栏位价格 为该用户对该栏位曝光制定商品的竞价。
\newline
}

\cventry{2014-2015}
{搜索部门}
{Python, Sql}{}
{同程旅游}
{
这个部门给网站的搜索接口建立搜索索引和优化排序接口。
\newline 我的工作:
\newline 1.使用sql和自定义方法在分布式存储的日志上分析各种过滤选项的转化率漏斗,找出转化率瓶颈。
\newline 2.优化过滤选项的设置和页面布局。
\newline 3.通过训练LR模型来拟合CTR,同事采用一些人过规则,来优化CTR。
\newline 4.通过不同优化版本和规则间的分流A,B来分析优化的结果。
\newline
}

\cventry{2012-2013}
{CIOD}
{Shell, Sql, Aix}{}
{IBM}
{
这个项目主要是负责给东京海上火灾保险公司开发和维护ERP系统的。
\newline My main duties:
\newline 1.在AIX操作系统上建立和维护多个测试环境。
\newline 2.在Jenkins平台上建立一套持续集成的打包发布工具。
\newline 3.备份,恢复和同步不同环境下的oracle数据库。
\newline
}

\section{Skills}
\cventry{语言}{C/C++, Scala, Python, Shell}{}{}{}{}
\cventry{操作系统}{Windows, Linux, OS X(日常使用)}{}{}{}{}
\cventry{机器学习}{分类, 聚类, deep learning(略懂)}{}{}{}{}
\cventry{版本控制}{Git, Svn}{}{}{}{}
\cventry{其他}{算法, 数据结构, VIM}{}{}{}{}
\closesection{}                   % needed to renewcommands
\renewcommand{\listitemsymbol}{-} % change the symbol for lists

\end{document}
