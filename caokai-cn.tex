\documentclass[11pt,a4paper]{moderncv}

% moderncv themes
%\moderncvtheme[blue]{casual}                 % optional argument are 'blue' (default), 'orange', 'red', 'green', 'grey' and 'roman' (for roman fonts, instead of sans serif fonts)
\moderncvtheme[green]{classic}                % idem
\usepackage{xunicode, xltxtra}
\XeTeXlinebreaklocale "zh"
\widowpenalty=10000

%\setmainfont[Mapping=tex-text]{文泉驿正黑}

% character encoding
%\usepackage[utf8]{inputenc}                   % replace by the encoding you are using
\usepackage{CJKutf8}

% adjust the page margins
\usepackage[scale=0.8]{geometry}
\recomputelengths                             % required when changes are made to page layout lengths
\setmainfont[Mapping=tex-text]{Microsoft YaHei}
\setsansfont[Mapping=tex-text]{Microsoft YaHei}
\CJKtilde

% personal data
\firstname{曹凯}
\familyname{}
%\title{}               % optional, remove the line if not wanted
\mobile{13818141545}                    % optional, remove the line if not wanted
\email{ck89119@gmail.com}                      % optional, remove the line if not wanted
%% \quote{\small{``Do what you fear, and the death of fear is certain.''\\-- Anthony Robbins}}

\nopagenumbers{}

\begin{document}
\maketitle

\section{教育}
\cventry{2012--2016}{硕士}{上海交通大学-电子信息与电气工程学院,导师:谷大武}{}{}{}
\cventry{2008--2012}{本科}{苏州大学-计算机科学与技术学院}{}{}{}

%\section{实习经历}
%\cventry{2014}{阿里巴巴}{}{}{}{}
%\cventry{2011}{中创软件工程有限公司}{}{}{}{}

\section{社区}
\cventry{GitHub}{github.com/ck89119}{}{}{}{}

\section{项目经历}
\cventry{2015}
{输入法引擎测试套件}
{Shell, Python}{}
{触宝}
{
这是触宝输入法引擎在Linux平台上的完整测试模块。从功能上,我们的引擎分为西方语言和中文两部分,因此我们的测试也分为两部分。
\newline 我在西方语言(包括英语,法语,德语等11种语言)方面的工作:
\newline 1. 编写脚本,从收集的用户数据中生成correction-test测试样例。
\newline 2. 编写脚本,为我们字典中每个一元对/二元对/三元对生成exact-word-test测试样例。
\newline 3. 自动化西方语言测试,并将其集成到测试框架中。
\newline
\newline 我在中文(包括拼音,注音,仓颉,笔画和五笔)方面的工作:
\newline 1. 编写脚本,为我们字典中每个一元对/二元对/三元对生成exact-word-test测试样例。
\newline 2. 将之前留下的平均排名,一选概率等脚本集成进中文测试中。
\newline 3. 自动化中文测试,并将其集成到测试框架中。
\newline
%It is a full test suite for CooTek keyboard engine on Linux platform.
%Our engine seperates western languages and Chinese apart in function, so this test suite has two parts as well.
%\newline My contributions in western language(English, French, German and so on, about 11 languages in total) part,
%\newline 1.Wrote scripts to generate correction-test cases from user input data which was collected by my teammate.
%\newline 2.Wrote scripts to generate exact-word-test cases for every unigram/bigram/trigram in our dictionaries.
%\newline 3.Automatized western language test part and integrated it into test suite.
%\newline My contributions in Chinese(Pinyin, Zhuyin, Cangjie, Bihua and Wubi) part,
%\newline 1.Wrote scripts to generate exact-word-test cases for every word in our dictionaries.
%\newline 2.Integrated scripts which were used for testing average rank, first rank percentage and sentence percentage.
%\newline 3.Automatized Chinese test part and integrated it into test suite.
%\newline
}

\cventry{2015}
{在Android app中增加性能测试功能}
{Java, Android}{}
{触宝}
{
为了模拟真实的测试环境,我们写了一个app来测试输入法各方面的表现。我的工作:
\newline 1. 编写测试计时模块及一系列派生的功能测试计时模块。
\newline 
%In order to simulate the real condition, we wrote an Android app for testing. My contributions,
%\newline 1.Wrote a time testing module and a series derivative testing modules.
%\newline
}

\cventry{2014}
{主流跑分app的opcode统计}
{Dalvik VM, Android}
{实习项目}
{阿里巴巴}
{
这个项目的目的是统计一个app运行时它所调用的每一个opcode的调用次数。我的工作:
\newline 1. 确定统计计数的时间点。
\newline 2. 通过强制转换到ALT模式来优化计数过程。
\newline 3. 利用USER\_SIGNAL输出统计结果。
\newline 
%The purpose of this project was to count the times each opcode is called when an Android app runs. My contributions,
%\newline 1.Designed the whole architecture.
%\newline 2.Determined the counting timing and the position of counting field.
%\newline 3.Figured out the idea, switching to ALT-mode forcely, to optimize counting process.
%\newline 4.Output the counting result by exploiting USER\_SIGNAL.
%\newline
}

\cventry{2012-2014}
{LoCCS功耗分析平台}
{JNA, 旁路攻击}
{研究项目}{}
{
这个项目是上海交通大学密码与计算机安全实验室(LoCCS)国家973项目的一部分。这个分析平台能覆盖旁路攻击一整套攻击流程中所有的步骤。我的工作:
\newline 1. 设计了最初版的UI,并用Java实现。
\newline 2. 测试JNA和JNI,即测试Java调用C++的可行性。
\newline 3. 编写的一系列的DES算法分析模块,例如:相关性分析,已知密钥分析,二阶分析。
\newline 
%This project was a part of National ’973’ Project in Lab of Cryptology and Computer Security (LoCCS). This platform was designed to cover all operations during an SCA. My contributions,
%\newline 1.Designed the origin UI of the software and implemented it in Java.
%\newline 2.Tested JNA and JNI for Java calling C++ libraries.
%\newline 3.Wrote a series of analytic modules of DES algorithm, e.g. CorrelationAnalysis, KnownKeyAnalysis, SecondOrderAnalysis.
%\newline
}

\cventry{2012.5}
{高效构造二叉树的研究}
{算法}
{研究项目}{}
{
这是我的本科毕设项目,毕设的目标是:利用二叉树的两种遍历顺序或者节点信息,高效地还原一整棵二叉树。我的工作:
\newline 1. 发明一种算法,通过二叉树的后序遍历和每个节点的第一个子节点构造二叉树。
\newline 2. 发明一种算法,通过二叉树的后序遍历和每个节点的下一个兄弟节点构造二叉树。
\newline 3. 构造一棵大约5000个节点的随机二叉树用于测试。
\newline
%This was my graduation project for undergraduate study. The purpose of this project was to find efficient algorithms that could construct a binary tree from the two kinds of traversal sequence and information of each node. My contributions,
%\newline 1.Found a constructing algorithm from postorder traversal and the first child of each node.
%\newline 2.Found a constructing algorithm from postorder traversal and nextsibling of each node.
%\newline 3.Created random binary trees(about 5000 nodes) for testing.
%\newline
}

%\section{奖项}
%\cventry{2012}{研究生入学二等奖学金}{}{}{}{}
%\cventry{2010}{苏州大学程序设计竞赛第二名}{}{}{}{}
%\cventry{2009}{ACM/ICPC亚洲区武汉赛区Honor Reward}{}{}{}{}
%\cventry{2009}{苏州大学人民奖学金一等奖,朱敬文奖学金,中创奖学金}{}{}{}{}
%\cventry{2007}{全国青少年信息学(计算机)奥林匹克分区联赛二等奖}{}{}{}{}

\section{Skills}
\cventry{语言}{C/C++, Java, Python, Shell}{}{}{}{}
\cventry{操作系统}{Windows, Linux, OS X(日常使用)}{}{}{}{}
\cventry{版本控制}{Git, Svn}{}{}{}{}
\cventry{Android}{VM, Java Dev}{}{}{}{}
\cventry{其它}{算法/数据机构, Vimer$\sim$}{}{}{}{}
\closesection{}                   % needed to renewcommands
\renewcommand{\listitemsymbol}{-} % change the symbol for lists

\end{document}
